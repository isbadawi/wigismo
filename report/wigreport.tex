\documentclass{WigReport}
\usepackage{epsfig} % so you can include .eps figures
\usepackage{alltt}  % so you can include program text 
\title{Wig Compiler Report}
\ReportNumber{2011-d} % replace  yyyy with a year and xx with your group #
\author{Ismail Badawi \\  % put your names here
        Carla Morawicz}
\begin{document}

\MakeTitlePage
\tableofcontents
%\listoffigures % comment out if you have no figures
%\listoftables  % comment out if you have no tables
\clearpage

% Replace each subsection below with the appropriate contents.
% Feel free to add more (sub)sections if you want. Write clearly
% and concisely. Assume that your readers know all about WIG and
% the COMP-520 course; don't quote the project description.
%
\section{Introduction}
\subsection{Clarifications}
Several aspects of the WIG language are unclear or ambiguous, many of them
of minor importance. The big issues mainly concern the type-correctness of
various language constructs. Chief amongst these issues is the behavior of
tuples and schemas; when are two tuples equivalent, do tuples necessarily
need to adhere to a declared schema, what exactly are the semantics of the 
three tuple operations, and so on. 

The {\tt show}, {\tt exit}, {\tt plug} and {\tt receive} constructs can 
also be treated with varying levels of strictness (and the provided
compilers do implement them in different ways). What types is a hole 
allowed to be plugged with (only strings, or can the other types be coerced
to strings)? What types can be received from {\tt show} or {\tt exit} 
statement? Must all holes be plugged? What happens if a hole is plugged
twice? Or if we try to plug a missing hole?

Our general approach has been very permissive. For tuples, we consider
structural equivalence; two tuples are type-equivalent (and can be assigned
to each other) if every field that appears in one appears in the other, and
with the same type. In particular, schema names, along with the order in
which the fields are given, are ignored.

When it comes to templating, we usually opt for silent failures. Holes can
be plugged twice; we use the rightmost value. Gaps can be omitted, empty
strings are used instead. A hole can be plugged by any type, including
tuples -- we provide a suitable string representation. Tuples cannot be
received, however -- parsing tuples from an HTML input would not be out of
the question, but doesn't seem useful from the end user's perspective.
We don't allow plugging missing holes; there doesn't appear to be any 
situation in which this is useful.

\subsection{Restrictions}
Our version of WIG has one restriction; {\tt show} and {\tt exit}
statements are not allowed inside functions. In practice, this probably
would not have terribly difficult to implement, considering we're able to
do it for sessions (there are just a couple of big picture complications,
chiefly relating to how function arguments are handled), but we didn't
plan for it from the beginning, and once we got down to it it didn't seem
all that important, considering exactly zero of the past benchmarks make
use of this feature.

\subsection{Extensions}
Aside from minor syntactic sugar like standard increments and {\tt for}
loops, we haven't made any extensions to our version of WIG.

\subsection{Implementation Status}
Our WIG implementation is feature-complete; it is able to generate working
WIG services for nearly all of the past benchmarks. Notably, it is able
to compile our benchmark ({\tt hangman.wig}), which previously only
{\tt lkwig} could compile.

\clearpage

\section{Parsing and Abstract Syntax Trees}
\subsection{The Grammar}

Our {\tt bison} grammar follows.

\makebox[\textwidth]{\hrulefill}
\begin{verbatim}

%right '=' tPLUSEQ tMINEQ tMULTEQ tDIVEQ tMODEQ tANDEQ tOREQ
%left tOR
%left tAND 
%nonassoc tEQ tLEQ tGEQ tNEQ '<' '>'
%left '+' '-'
%left '*' '/' '%'
%left tKEEP tDISCARD
%right tCOMBINE
%right '!' tUMINUS tINC tDEC

%%

service : tSERVICE '{' htmls schemas nevariables functions sessions '}'
        | tSERVICE '{' htmls schemas functions sessions '}'          
;

htmls : html 
        | htmls html
;

html : tCONST tHTML tIDENTIFIER '=' tHTMLOPEN nehtmlbodies tHTMLCLOSE ';'
     | tCONST tHTML tIDENTIFIER '=' tHTMLOPEN tHTMLCLOSE ';'             
;

nehtmlbodies : htmlbody 
             | nehtmlbodies htmlbody
;

htmlbody : '<' tIDENTIFIER attributes '>'
         | '<' tIDENTIFIER attributes tCLOSINGTAG
         | tOPENINGTAG tIDENTIFIER '>'
         | tOPENINGGAP tIDENTIFIER tCLOSINGGAP
         | tWHATEVER
         | tMETA
         | '<' tINPUT inputattrs '>'
         | '<' tSELECT inputattrs '>' nehtmlbodies tOPENINGTAG tSELECT '>'
         | '<' tSELECT inputattrs '>' tOPENINGTAG tSELECT '>'        
;

inputattrs : inputattr 
           | inputattrs inputattr
;

inputattr : tNAME '=' attr
          | tTYPE '=' inputtype
          | attribute
;

inputtype : tTEXT 
          | tRADIO

attributes : /* empty */ 
           | neattributes
;

neattributes : attribute 
             | neattributes attribute
;

attribute : attr 
          | attr '=' attr
;

attr : tIDENTIFIER 
     | tSTRINGCONST
;

schemas : /* empty */ 
        | neschemas
;

neschemas : schema 
          | neschemas schema
;

schema : tSCHEMA tIDENTIFIER '{' fields '}'
;

fields : /* empty */ 
       | nefields
;

nefields : field 
         | nefields field
;

field : simpletype tIDENTIFIER ';'
;

nevariables : variable 
            | nevariables variable 
;

variable : type identifiers ';'
;

identifiers : tIDENTIFIER 
            | identifiers ',' tIDENTIFIER
;

simpletype : tINT 
            | tBOOL 
            | tSTRING 
            | tVOID
;

type : simpletype 
     | tTUPLE tIDENTIFIER
;

functions : /* empty */ 
          | nefunctions
;

nefunctions : function 
            | nefunctions function
;

function : type tIDENTIFIER '(' arguments ')' compoundstm
;

arguments : /* empty */ 
          | nearguments
;
nearguments : argument 
            | nearguments ',' argument
;

argument : type tIDENTIFIER
;

sessions : session 
         | sessions session
;

session : tSESSION tIDENTIFIER '(' ')' compoundstm
;

stms : /* empty */ 
     | nestms
;
nestms : stm 
       | nestms stm
;
stm : ';'
    | tSHOW document receive ';'
    | tEXIT document ';'
    | tRETURN ';'
    | tRETURN exp ';'
    | tIF '(' exp ')' stm
    | tIF '(' exp ')' stmnoshortif tELSE stm
    | tWHILE '(' exp ')' stm
    | tFOR '(' optionalexp ';' optionalexp ';' optionalexp ')' stm
      { EXP* condition = $5 != NULL ? $5 : makeEXPboolconst(1);
        STATEMENT *body = $7 != NULL ? makeSTATEMENTsequence($9, makeSTATEMENTexp($7)) : $9;
        STATEMENT *loop = makeSTATEMENTwhile(condition, body);
        $$ = $3 != NULL ? makeSTATEMENTsequence(makeSTATEMENTexp($3), loop) : loop;
      } 
    | tUNTIL '(' exp ')' stm
    | compoundstm
    | exp ';'
;

stmnoshortif : ';'
    | tSHOW document receive ';'
    | tEXIT document ';'
    | tRETURN ';'
    | tRETURN exp ';'
    | tIF '(' exp ')' stmnoshortif tELSE stmnoshortif
    | tWHILE '(' exp ')' stmnoshortif
    | tUNTIL '(' exp ')' stmnoshortif
    | tFOR '(' optionalexp ';' optionalexp ';' optionalexp ')' stmnoshortif
      { EXP* condition = $5 != NULL ? $5 : makeEXPboolconst(1);
        STATEMENT *body = $7 != NULL ? makeSTATEMENTsequence($9, makeSTATEMENTexp($7)) : $9;
        STATEMENT *loop = makeSTATEMENTwhile(condition, body);
        $$ = $3 != NULL ? makeSTATEMENTsequence(makeSTATEMENTexp($3), loop) : loop;
      } 
    | compoundstm
    | exp ';'
;

document : tIDENTIFIER 
         | tPLUG tIDENTIFIER '[' plugs ']'
;
receive : /* empty */
        | tRECEIVE '[' inputs ']'
;

compoundstm : '{' nevariables stms '}' 
            | '{' stms '}'            
;

plugs : plug 
      | plugs ',' plug
;

plug : tIDENTIFIER '=' exp
;

inputs : /* empty */ 
       | neinputs
;

neinputs : input 
         | neinputs ',' input
;

input : lvalue '=' tIDENTIFIER
;

exp : lvalue
    | lvalue '=' exp
    | lvalue tPLUSEQ exp
    | lvalue tMINEQ exp
    | lvalue tMULTEQ exp
    | lvalue tDIVEQ exp
    | lvalue tMODEQ exp
    | lvalue tANDEQ exp
    | lvalue tOREQ exp
    | tINC lvalue
    | tDEC lvalue
    | exp tEQ exp
    | exp tNEQ exp
    | exp '<' exp
    | exp '>' exp
    | exp tLEQ exp
    | exp tGEQ exp
    | '!' exp
    | '-' exp %prec tUMINUS
    | exp '+' exp
    | exp '-' exp
    | exp '*' exp
    | exp '/' exp
    | exp '%' exp
    | exp tAND exp
    | exp tOR exp
    | exp tCOMBINE exp
    | exp tKEEP tIDENTIFIER
    | exp tKEEP '(' identifiers ')' 
    | exp tDISCARD tIDENTIFIER        
    | exp tDISCARD '(' identifiers ')'
    | tIDENTIFIER '(' exps ')'
    | tINTCONST
    | tBOOLCONST
    | tSTRINGCONST
    | tTUPLE '{' fieldvalues '}'
    | '(' exp ')'                 
;

optionalexp : /* empty */
            | exp

exps : /* empty */ 
     | neexps
;

neexps : exp 
       | neexps ',' exp
;

lvalue : tIDENTIFIER 
       | tIDENTIFIER '.' tIDENTIFIER
;

fieldvalues : /* empty */ 
            | nefieldvalues
;

nefieldvalues : fieldvalue 
              | fieldvalues ',' fieldvalue
;

fieldvalue : tIDENTIFIER '=' exp
;

\end{verbatim}
\makebox[\textwidth]{\hrulefill}

\subsection{Using the {\tt flex} or {\tt SableCC} Tool}

Our scanner includes the following start conditions:

\begin{itemize}

\item WIG -- The initial state, where all WIG keywords are reserved.

\item HTML -- Entered after {\tt <html>} is scanned and left when 
              {\tt </html>} is scanned. We handle meta tags properly so 
              long as they do not end with a -; that is, 
              {\tt <!-- is fine -->}, but {\tt <!-- isn't --->}, for 
              simplicity; correctly recognizing the second would require a 
              more complicated action in the same vein as the one for block
              comments, but we also want in this case to store the content
              of the tag, and that makes things less straightforward.

\item HTMLTAG -- Entered from HTML when a {\tt <} or {\tt </} is scanned, 
                 and left when {\tt >} or {\tt />} is scanned. Here 
                 {\tt input} and {\tt select} are reserved words.

\item HTMLINPUT -- Entered from HTMLTAG if the tag's name is {\tt input}.

\item TYPERHS -- Entered from HTMLINPUT when {\tt type=} is scanned. In
                 this state only {\tt "text"}, {\tt text}, {\tt "radio"} 
                 and {\tt radio} can be scanned. This sequence of states 
                 (HTMLTAG, HTMLINPUT, TYPERHS) is needed because other tags
                 (notably, {\tt <link>}) may have a {\tt type} attribute 
                 that takes other values.

\item HTMLRHS -- Entered from HTMLTAG when a {\tt =} is scanned, and left 
                 after either a string constant is scanned, or an 
                 identifier.  "Identifiers" here can start with a number, 
                 so that e.g.  {\tt size=20} is valid.

\item INPUTRHS -- same as HTMLRHS, but knows to return to HTMLINPUT instead
                  of HTMLTAG.

\item GAP -- Entered from HTML when {\tt <{[}} is scanned and left when 
       {\tt {]}>} is scanned.  (We played around with the provided 
       compilers and found that gaps are usually not recognized within 
       tags.) Here nothing is reserved.  

\end{itemize}

Some fairly subtle bugs were found to have been caused by our initially
naive use of start conditions. For example, we had a case where

{\tt <input name="test" type="text">}

would parse properly, but

{\tt <input type="text" name="test">}

would cause a syntax error. Our flex scanner contains a fair amount of
code duplication as a result, as some rules appear several times in
different states, the only difference being what state they begin in their 
associated action. There doubtless exist nicer ways of dealing with this 
sort of thing.

\subsection{Using the {\tt bison} or {\tt SableCC} Tool}
The provided start grammar included 273 shift/reduce conflicts. Most of
them (that is, all but one) were resolved by specifying operator
precedences using {\tt \%left}, {\tt \%right} and {\tt \%nonassoc} 
directives.  One was caused by the "dangling else" problem discussed in the
readings; taking a cue from the JOOS compiler, we introduced a 
{\tt stmnoshortif} production for the {\tt then} part of an 
{\tt if ...  else} statement.

\subsection{Abstract Syntax Trees}
Our abstract syntax trees are composed of 18 node types, namely
{\tt SERVICE}, {\tt SESSION}, {\tt HTML}, {\tt HTMLBODY}, {\tt ATTRIBUTE}, 
{\tt PLUG}, {\tt RECEIVE}, {\tt INPUT}, {\tt SCHEMA}, {\tt ID}, 
{\tt VARIABLE}, {\tt FUNCTION}, {\tt TYPE}, {\tt STATEMENT}, 
{\tt DOCUMENT}, {\tt EXP}, {\tt ARGUMENT}, and {\tt FIELDVALUE}, all mapping
straightforwardly to elements of the WIG language.

\subsection{Desugaring}
We provide the following constructs, implemented as syntactic sugar:

\begin{itemize}
\item {\tt for} and {\tt until} loops, which are transformed into 
equivalent while loops
\item The compound assignment operators {\tt +=}, {\tt -=}, {\tt *=},
{\tt /=}, {\tt \%=}, {\tt \&\&=} and {\tt ||=}, which are transformed into
equivalent assignments
\item Standard prefix increment and decrement operators, which are 
transformed into equivalent assignments (we don't provide the postfix
versions; typically those change the value and return the old one, and
there isn't really a way to do that in WIG, so implementing them as
syntactic sugar would be difficult, to say the least).
\end{itemize}

Each of these corresponds to an extra token recognized by the scanner,
and one or two extra productions in the parser. For instance, {\tt until}
loops are implemented as simply as:

\begin{verbatim}
stm : ...
    | tUNTIL '(' exp ')' stm
      { $$ = makeSTATEMENTwhile(makeEXPnot($3), $5); }
      ...
\end{verbatim}

\subsection{Weeding}
The following cases are weeded out (straightforwardly, by traversing the
AST):

\begin{itemize}
\item variables of type {\tt void}, in any context (i.e. as arguments to
functions, as global variables, as local variables in sessions or
functions, and inside schemas.)
\item sessions that don't always {\tt exit}
\item {\tt void} functions where something is returned
\item non {\tt void} functions where nothing is returned
\item {\tt <input>} tags without {\tt name} or {\tt type} attributes
\item {\tt <select>} tags without a {\tt name} attribute
\item {\tt return} statements in sessions
\item {\tt exit} and {\tt show} statements in functions
\end{itemize}

There are a few cases that we considered weeding but ultimately decided not
to (in keeping with our general permissive approach):

\begin{itemize}
\item empty schema declarations, e.g. {\tt schema S \{\}}. This is never
useful, but we don't really the harm in it.
\item {\tt <input>} tags with more than one name, or more than one type.
This doesn't really cause trouble because the order in which we traverse
the AST is always the same, so even though there's more than one name or
type, it's actually well defined which one we'll use.
\item {\tt <select>} tags with more than one name, similarly
\end{itemize}

\subsection{Testing}
The validity of the scanning and parsing phase is verified by checking
whether {\tt pretty(parse(x)) == pretty(parse(pretty(parse(x))))} holds
for every WIG service x. This is obviously undecidable, so in practice we
just tried for many many WIG services. A representative sample is included
in the {\tt group-d/wig/tests} directory, and this property is checked for
these during the build process.

\clearpage

\section{Symbol Tables}
\subsection{Scope Rules}
There is one global scope; in particular, functions, HTMLs, schemas,
global variables and sessions all reside in the same namespace, so their
names may not clash ({\tt wigismo}'s error messages might be slightly
misleading in this respect). New scopes are introduced by blocks, which
can occur anywhere, bringing with them variable declarations. Variable
names in nested scopes shadow those in outer scopes. A variable name may
not be reused inside a single scope.

\subsection{Symbol Data}
Each {\tt SYMBOL} just keeps a pointer to the relevant AST node. From there
we can get at any information we deem relevant, such as the line number on
which the symbol appears, its type if it's a variable or an argument, its
return type if it's a function, and so on.

\subsection{Algorithm}
We implement the cactus stack structure for our symbol tables as seen in 
class. There is a global symbol table for the whole service ({\tt mst}, 
declared in {\tt symbol.h}); in addition, whenever a block statement is 
encountered (that is, whenever a new scope is entered, because blocks are 
the only way to introduce new scopes in WIG), the node for that statement 
keeps a pointer to its local symbol table, which keeps a pointer to the
symbol table of the next outer scope. First, we insert all html literals,
schemas, global variables, functions and sessions into the {\tt mst}. Then
we traverse the AST, keeping track of which scope we're in. When a new
scope is entered, all of its local variable declarations are inserted into
the table for that statement ({\tt blockS.table}), and in so doing are
checked for uniqueness. Whenever an identifier is used, we check the
current scope and work our way up to the {\tt mst} looking for a match,
reporting an error if there isn't one.

\subsection{Testing}
The symbol table phase was tested by constructing a minimal WIG service
for each possible sort of error, and running wigismo on each of them,
checking that an appropriate error message is printed, and compilation is
terminated. These test cases all live in {\tt group-d/wig/tests/symbol},
and cover the following situations:

\begin{itemize}
\item Calling undeclared functions
\item Using the same name for different arguments to a function
\item Using the same name for local variables in the same scope
\item Using the same name for two global entities
\item Plugging an undeclared hole
\item Showing or exiting with an undeclared HTML
\item Receiving an undeclared input
\item Receiving an input into an undeclared variable or undeclared field 
of a tuple
\item Declaring a tuple with an undeclared schema
\item Assigning to an undeclared variable, or undeclared field of a tuple
\item Using undeclared variables or undeclared fields of tuples in
expressions
\end{itemize}

Further, wigismo's pretty printer can be instructed to dump the contents
of the symbol table whenever a new scope is entered. Each new scope is
marked by an extra level of indentation. This allows us to compare the
results of the symbol table phase against a manual construction. For
instance, the following (rather contrived) service:

\begin{verbatim}
service {
    const html Test = <html>Test</html>;
    schema S {
        int x;
    }
    int x;
    int test(bool x) {
        int y;
        y = 10;
        while (y > 0) {
            string x;
            y = y - 1;
        }
        return 0;
    }
    session Main() {
        tuple S x;
        x.x = test(true);
        exit Test;
    }
}
\end{verbatim}

produces the following pretty-printer output:

\begin{verbatim}
service
{
    =================================================
    =============|Symbols in this Scope|=============
    =================================================
     S: schema
     x: int variable
     test: function returning int
     Test: html
     Main: session
    =================================================

    const html Test = <html>Test</html>;

    schema S {
        int x;
    }

    int x;
    int test(bool x)
    {
        =================================================
        =============|Symbols in this Scope|=============
        =================================================
         S: schema
         x: int variable
         test: function returning int
         Test: html
         Main: session
             x: bool argument
             y: int variable
        =================================================

        int y;
        y = 10;
        while (y > 0)
        {
            =================================================
            =============|Symbols in this Scope|=============
            =================================================
             S: schema
             x: int variable
             test: function returning int
             Test: html
             Main: session
                 x: bool argument
                 y: int variable
                     x: string variable
            =================================================

            string x;
            y = (y - 1);
        }
        return 0;
    }

    session Main()
    {
        =================================================
        =============|Symbols in this Scope|=============
        =================================================
         S: schema
         x: int variable
         test: function returning int
         Test: html
         Main: session
             x: tuple S variable
        =================================================

        tuple S x;
        x.x = test(true);
        exit Test;
    }
}
\end{verbatim}

\clearpage

\section{Type Checking}
\subsection{Types}
The WIG language has support for three primitive types, namely {\tt int},
{\tt bool}, and {\tt string}. It also has tuples, which are collections
of key-value pairs adhering to a {\tt schema}.

\subsection{Type Rules}
\begin{itemize}
\item A {\tt service} is type correct if the sessions and functions within
      are type correct.
\item A session or function is type correct if its body is type correct.
\item The skip statement ({\tt ;}) is always type correct.
\item A sequence {\tt s1, s2} of statements is type correct if {\tt s1} and 
{\tt s2} are both type correct.  
\item The expressions plugged into a {\tt show} statement must type check; 
the variables received from a {\tt show} must not be of a tuple type.
\item The expressions plugged into an {\tt exit} statement must type check. 
\item A {\tt return} statement is type correct if the type of the 
      expression returned is compatible with the return type of the 
      function. (Return statements in sessions do not pass the weeder.)
\item Block statements are type correct if their body is type correct.
\item An {\tt if} statement is type correct if the condition expression is
      of type {\tt bool} and the body is type correct.
\item An {\tt if ... else} statement is type correct if the condition 
      expression is of type {\tt bool} and both the then part and the else 
      part of the statement are type correct.
\item A {\tt while} statement is type correct if the condition expression 
      is of type {\tt bool} and the body is type correct.
\item An expression statement is type correct if the expression is type 
      correct.
\item The left and right operands of the arithmetic opreators {\tt -},
      {\tt *}, {\tt /}, and {\tt \%} must be of type {\tt int}; the result 
      is of type {\tt int}.
\item For addition, the two operands must be of type {\tt string} or 
      {\tt int}. If either or both are {\tt string}s, the result is a 
      {\tt string}, otherwise it's an {\tt int}.
\item The type of the expression on the right hand side of a {\tt =} must 
      either be of the same type or a compatible type as the identifier on 
      the left hand side; the result is the type that is evaluated from the
      right-hand side expression.
\item The left and right operatands of the boolean operators {\tt \&\&} and
      {\tt ||} must be of type {\tt bool}; the result is of type {\tt bool}.
\item The left and right operands of the relation operators {\tt <}, 
      {\tt >}, {\tt <=}, {\tt >=} must be of type {\tt int}; the result is
      of type {\tt int}.
\item For the equality operators {\tt ==} and {\tt !=}, the types of the
      left and right operands must be the same or compatible, where
      compatibility in the case of tuples refers to structural equivalence 
      -- that is, every variable that appears in one of the schemas also 
      appears in the other schema with the same type -- the result is of
      type {\tt bool}.
\item For the unary minus operator, the expression being negated must have
      type {\tt int}; the result is of type {\tt int}.
\item For the negation operator {\tt !}, the expression being negated must
      type {\tt bool}; the result is of type {\tt bool}.
\item The types of the arguments of a function call must correspond to the
      types declared in the function's signature; the type of the 
      expression is the return type of the function.
\item Any integer constant (matching the pattern
      {\tt 0|{[}1-9{]}{[}0-9{]}*}) has type {\tt int}.
\item The literals {\tt true} and {\tt false} are of type {\tt bool}. 
\item Any double quoted string is of type {\tt string}.  
\item Tuple literals have a tuple type -- their schema is defined 
      implicitly by the types of their field values, assuming they type 
      check.
\item For the keep and discard operators ({\tt \textbackslash +} and 
      {\tt \textbackslash -}), the left operand must be of type tuple; each 
      identifier appearing on the right hand side must appear in its 
      schema; the result is tuple type with a new schema generated 
      automatically. 
\item For the combine operator ({\tt <<}), the left and right operands must
      be of type tuple, the common variables in their schemas must have the
      same type, and the result is a tuple whose schema is composed of the
      union of their variables.
\end{itemize}

\subsection{Algorithm}
Type checking is done by traversing the AST and storing the computed type 
of each expression in the {\tt TYPE} field of the corresponding {\tt EXP}
AST node. This process is largely straightforward; to constants and
literals we can immediately associate a type, for identifiers we simply
read the information gathered during the symbol table phase, and from 
there the type of each expression can be obtained in accordance with the
type rules. Once each expression has been annotated with its type, we have
enough information to enforce the type rules for statements, functions,
sessions, and finally the entire service.

\subsection{Testing}
As in the symbol table phase, a suite of minimal WIG services that provoke
every possible error message is provided. These live in
{\tt group-d/wig/tests/type}, and cover the following situations:

\begin{itemize}
\item Combining incompatible tuples
\item Discarding or keeping a missing field from a tuple
\item Using anything but a boolean variable in the condition of an
      {\tt if} or {\tt while} statement
\item Comparing non-comparable types
\item Assigning to a variable the result of an expression whose type it is
      not assignable from
\item Receiving tuples
\item Returning from a function an expression whose type is not compatible
      with the function's return type
\item Using the arithmetic operators (except {\tt +}) with anything but
      integers
\item Using the relational operators with anything but booleans
\item Calling functions with incompatible argument types
\end{itemize}

Also as in the symbol phase, wigismo's pretty-printer was enhanced to
annotate every expression with its type. Anonymous tuples are given unique
generated schema names. The service above produces the following output:

\begin{verbatim}

service
{
    const html Test = <html>Test</html>;

    schema S {
        int x;
    }

    int x;
    int test(bool x)
    {
        int y;
        (y = (10 : int) : int);
        while (((y : int) > (0 : int) : bool))
        {
            string x;
            (y = ((y : int) - (1 : int) : int) : int);
        }
        return (0 : int);
    }

    session Main()
    {
        tuple S x;
        (x.x = (test((true : bool)) : int) : int);
        exit Test;
    }
}
\end{verbatim}

\clearpage

\section{Resource Computation}
\subsection{Resources}
We assign unique integer identifiers to every {\tt if}, {\tt if ... else},
and {\tt while} statement; these are used to generate appropriate function
names (as described in the code generation section) and are analogous to
labels for {\tt goto} statements in jasmin.

We also assign unique integer ids to every variable. There isn't a clean
way to introduce arbitrary blocks in python, so scoping can be awkward.
Even if we opt for an ugly solution (e.g. use {\tt if True:} to introduce
blocks), we still have to deal with the odd control flow induced by
{\tt show} statements. Making every variable unique in this way makes
saving local state very easy -- we can just have a big dictionary of local
variables, since every key is unique (in fact, we could dispense with the
variable names in the generated code entirely, but we keep them to make it
easier to read.)

\subsection{Algorithm}
We maintain a global counter, and traverse the AST. Whenever an {\tt if},
{\tt if ... else}, or {\tt while} statement is encountered, we increment
the counter and assign it to the statement.

Assigning ids to variables is done during the parsing phase; an
incrementing id is assigned whenever a {\tt VARIABLE} node is created.

\subsection{Testing}
There isn't very much to test here. This phase is necessary for the code
generation phase; verifying that this latter phase works is sufficient.

\clearpage

\section{Code Generation}
\subsection{Strategy}
{\tt wigismo} generates python code. All of the WIG language constructs, 
with the notable exception of {\tt show} and {\tt exit}, have equivalents
in python. A {\tt while} loop is a {\tt while} loop, a {\tt ||} operator is
an {\tt or} operator, and so on, but the semantics of {\tt show} makes for
less than straightforward control flow. The big picture is as follows:

\begin{itemize}
\item Every service defines two variables: {\tt g}, an instance of
{\tt wigismo.Store}, which stores global variables, and {\tt l}, a
dictionary which stores local variables. Variables are all given unique
integer ids, so this one dictionary is sufficient (that is, scoping is not
an issue). No initialization happens; every access to a variable, local
or global, is wrapped in a call to {\tt get}, with a suitable default
value based on type.
\item HTMLs become functions which print the contents of the HTML. Holes
are the arguments to the function, and are inserted into the output using
string interpolation. {\tt show} and {\tt exit} statements partly 
correspond to just calling these functions, passing any values plugged as 
arguments. (python supports both default arguments and keyword arguments;
holes are all given a default value of {\tt ''}, and the HTML functions
are always called with every argument named, so in this way holes can be
omitted easily.)
\item Schemas don't appear in the generated code at all.
\item Functions map to ordinary python functions. Since we don't allow
{\tt show} or {\tt exit} statements in functions, this is straightforward.
\item Sessions become sets of functions. Since python doesn't support the
{\tt goto} statement, every session is split into several functions, one
for every possible entry or jump point (the start of a loop, after a loop,
after an {\tt if}, the start of an {\tt else}, after an {\tt else}, and
after a {\tt show}). This is necessary because regular control flow is
often impossible to apply (e.g. what if we have a {\tt show} statement
inside a loop?)
\item When a {\tt show} statement is encountered, the locals dictionary
is written to a file, along with the name of the function to call when
this session is resumed (this function is called using reflection). If
any values are to be received, this happens at the beginning of that
function.
\item In some cases we have to be careful about generating valid code;
for instance, python contains many reserved keywords which aren't reserved
in WIG (a common one is {\tt pass}), so we append underscores to function
argument names and hole names to avoid conflicts. HTML literals end up in
python strings, so some characters (backslashes, single quotes, newlines)
have to be escaped.
\end{itemize}

\subsection{Code Templates}
Python has significant whitespace; blocks are marked by levels of
indentation. This makes certain ideas in the template difficult to show
(as some templates reset the indentation level). In what follows, take
leading whitespace to mean "some unspecified but nonzero level of
indentation".

\noindent
The {\tt exit} statement

{\tt exit plug D {[}g1=x1, ..., gn=xn{]}}

\noindent
has template:

\begin{verbatim}
    wigismo.output(sessionid, lambda: output_D(g1=x1, ..., gn=xn), exit=True)
    sys.exit(0)
\end{verbatim}

\noindent
The {\tt show} statement (with precomputed destination {\tt NEXT}):

{\tt show plug D [g1=x1, ..., gn=xn] receive [y1=i1, ..., yn=in]}

\noindent
has template:

\begin{verbatim}
    wigismo.output(sessionid, lambda: output_D(g1=x1, ..., gn=xn))
    state = wigismo.Store(sessionid)
    state.set('locals', l)
    state.set('start', NEXT)
    sys.exit(0)

# reset indentation

def NEXT(sessionid):
    y1 = wigismo.get_field('i1', type(y1))
    ...
    yn = wigismo.get_field('in', type(yn))
\end{verbatim}

\noindent
The skip statement:

    {\tt ;}

\noindent
has template

    {\tt pass}

\noindent
The {\tt return} statement:

    {\tt return E;}

\noindent
has template:

    {\tt return E}

\subsection{Algorithm}
Describe your algorithm for generating code.

\subsection{Runtime System}
Since {\tt wigismo} generates python code, we have access to a very large
standard library. The functionality we need has been encapsulated into the
{\tt wigismo.py} module, which provides the following facilities:

\begin{itemize}
\item the {\tt random\_string} function generates random strings, for use
in generating session identifiers (using python's {\tt random} module) 
\item the {\tt get\_field} function handles parsing and retrieving CGI 
fields, and coercing them to appropriate types (using python's {\tt cgi}
module)
\item the {\tt output} function wraps the output HTML in a form if
necessary (a neat use of higher order functions)
\item the {\tt Store} class provides thread-safe storage of key-value pairs
to a file, for use in maintaining global state, and dumping local state
following a {\tt show} statement
\item the {\tt tuple\_keep}, {\tt tuple\_discard} and {\tt tuple\_combine}
functions implement the three tuple operations
\item While assignments in python can be chained (e.g. {\tt x = y = 4}), a
statement like {\tt x = (y = 2) + 3} raises a {\tt SyntaxError}; that is,
assignments aren't really expressions in python. To get around this, the
{\tt set} function performs the assignment and returns the value assigned.
\end{itemize}

\subsection{Sample Code}
\begin{verbatim}
#!/usr/bin/python
import sys
import wigismo
import cgitb
cgitb.enable()

g = wigismo.Store('tiny_globals.pck')
l = {}

def output_Welcome():
    sys.stdout.write(' ')
    sys.stdout.write('<body >')
    sys.stdout.write('\n    Welcome!\n  ')
    sys.stdout.write('</body>')
    sys.stdout.write(' ')

def output_Pledge():
    sys.stdout.write(' ')
    sys.stdout.write('<body >')
    sys.stdout.write('\n    How much do you want to contribute?\n    ')
    sys.stdout.write('<input name="contribution" type="text" size="4" >')
    sys.stdout.write('\n  ')
    sys.stdout.write('</body>')
    sys.stdout.write(' ')

def output_Total(total_=''):
    sys.stdout.write(' ')
    sys.stdout.write('<body >')
    sys.stdout.write('\n    The total is now ')
    sys.stdout.write('%s' % total_)
    sys.stdout.write('.\n  ')
    sys.stdout.write('</body>')
    sys.stdout.write(' ')

def session_Contribute(sessionid):
    wigismo.set(l, 'i_2', 87)
    wigismo.output(sessionid, lambda: output_Welcome())
    state = wigismo.Store(sessionid)
    state.set('locals', l)
    state.set('start', 'session_Contribute_show_0')
    sys.exit(0)

def session_Contribute_show_0(sessionid):
    wigismo.output(sessionid, lambda: output_Pledge())
    state = wigismo.Store(sessionid)
    state.set('locals', l)
    state.set('start', 'session_Contribute_show_1')
    sys.exit(0)
    pass

def session_Contribute_show_1(sessionid):
    l['i_2'] = wigismo.get_field('contribution', int)
    g.set('amount_1', (g.get('amount_1', 0) + l.get('i_2', 0)))
    wigismo.output(sessionid, lambda: output_Total(total=g.get('amount_1', 0)), exit=True)
    sys.exit(0)
    pass

def wigismo_restart():
    state = wigismo.Store(wigismo.sessionid)
    l.update(state.get('locals'))
    globals()[state.get('start')](wigismo.sessionid)

sessions = ['Contribute']
for session in sessions:
    if wigismo.sessionid == session:
        globals()['session_%s' % session]('%s$%s' % (session, wigismo.random_string(20)))
        sys.exit(0)
    elif wigismo.sessionid.startswith('%s$' % session):
        wigismo_restart();
        sys.exit(0)

print 'Content-type: text/html'
print
print '<title>Illegal request</title>'
print '<p>You entered an invalid session name.</p>'
print '<p>Try one of these:</p>'
print '<ul>'
for session in sessions:
    print '<li><a href="?%s">%s</a></li>' % (session, session)
print '</ul>'
sys.exit(0)
\end{verbatim}

\subsection{Testing}
The code generation phase was tested by trying to compile and run several
services (past benchmarks, and services in the 
{\tt public\_html/wig/examples} directory). Notably, {\tt wigismo} was 
able to successfully compile our benchmark {\tt hangman.wig} service. We 
also compared its output to our manual construction of the {\tt 08rps.wig}
service during the CGI vs WIG milestone. Many somewhat interesting issues
surfaced while trying out various benchmarks:

\begin{itemize}
\item In python, blocks are marked by levels of indentation. A common 
pattern in WIG services (more so than in other languages, because WIG 
doesn't have arrays) is to have large if-else-if structures, like so:

\begin{verbatim}
if (c1) {
    ...
}
else if (c2) {
    ...
}
else if (c3) {
    ...
}
\end{verbatim}

There's nothing special about {\tt else if}; in the AST, we just have an
{\tt if ... else} statement whose {\tt else} part happens to be another
{\tt if} or {\tt if ... else} statement. As such, we were generating code
like this:

\begin{verbatim}
if c1:
    ...
else:
    if c2:
        ...
    else:
        if c3:
            ...
\end{verbatim}

The indentation level kept climbing. This might not seem like such a big
deal, except that, as it turns out, python has a limit to the number of
indented blocks one can have; this limit is hardcoded somewhere to be 100.
Some services (notably, {\tt 2008/wig/group-4/phish.wig}) broke this
limit, causing the generated code to raise an {\tt IndentationError: 
too many indentation levels.}.

The somewhat ugly fix to this was to explicitly check, when generating
code for an {\tt if ... else} statement, whether the else part was itself
an {\tt if} or {\tt if ... else} statement, and if so writing {\tt el}
instead of {\tt else:\textbackslash n}, and setting a special flag so that
the {\tt else} part knows not to indent itself, resulting in code like:

\begin{verbatim}
if c1:
    ...
elif c2:
    ...
elif c3:
    ...
\end{verbatim}
\item Blocks in python must contain at least one statement, otherwise an
{\tt IndentationError} error is raised. A snippet like this:

\begin{verbatim}
if (some_condition) {}
nextStatement();
\end{verbatim}

generated code like this:

\begin{verbatim}
if some_condition:
nextStatement()
\end{verbatim}

Less contrived examples involve reaching the end of a block after coming
back from a {\tt show} statement; this resulted in empty functions being
generated. The (again, somewhat ugly) fix to this was to explicitly end 
every block with a {\tt pass} statement, resulting for the above in code
like this:

\begin{verbatim}
if some_condition:
    pass
nextStatement()
\end{verbatim}
\end{itemize}

\clearpage

\section{Availability and Group Dynamics}
\subsection{Manual}
Running {\tt wigismo} on the input WIG service ({\tt example.wig}, say)
generates a python file ({\tt example.py}). This file has to be moved to
whatever directory CGI scripts are served from, and its permissions have
to be changed to {\tt 755}. The generated file depends on a python module 
we wrote called {\tt wigismo}; there could be trouble when it tries to
import it, depending on the environment. The simplest way to make it work
is to copy the {\tt wigismo.py} file into the {\tt cgi-bin} directory, or
into some directory on the python search path. Note that setting the
{\tt \$PYTHONPATH} environment variable is not reliable; CGI scripts
typically don't run under the same environment as the user. An ad-hoc
solution would be to alter the search path (in the python file, after
compiling), by adding this line just below {\tt import sys}:

{\tt sys.path.insert(0, '/absolute/path/to/wigismo.py')}

\subsection{Demo Site}
Demo services available at 
{\tt http://www.cs.mcgill.ca/\textasciitilde ibadaw/wigismo/}.

\subsection{Division of Group Duties}
The majority of the work was done through pair programming; that is,
both team members sitting on one computer, alternately typing. There are
several advantages to this. For one thing, both members of the team work
on every part of the compiler; this helps reduce any discrepancy in the
experience gained, and it also makes oversights and such less common, since
every piece of code has two pairs of eyes looking at it. Moreover, since
we have "joint ownership" of the code, there is no blame to be assigned
when something breaks. Having your partner next to you also makes you less
likely to procrastinate. 

Of course, this approach is not necessarily perfect. For instance, it
requires that everything be worked on together -- if one member is overly
busy, no progress can be made. It's also hard to quantify -- svn commits,
for instance, may not accurately reflect who actually committed what (and
there is no concept of a "joint commit" or anything like that). It can also
take longer -- in some cases, if the problem is well understood and it's
just a matter of tedious typing, it would definitely be faster to have
two programmers working independently.

\clearpage

\section{Conclusions and Future Work}
\subsection{Conclusions}
First, provide a brief summary of the report.  Next, describe the main
things that you learned in the course of this work, and attempt to
draw new conclusions, even if they are just ``soft'' experience-based
ones.  Detail the things you learned that you did not expect to learn.

\subsection{Future Work}
The WIG language is very restrictive; it is practically impossible to 
write something worthwhile in it. In particular, WIG needs:

\begin{itemize}
\item Support for arrays, or indeed any useful data structure. A majority
of the benchmarks end up using some large number of variables (or 
generating a whole bunch of code) to emulate this anyway. Depending on the
target language, this might not even be so difficult to implement; one
could think of transforming arrays in WIG into lists in python, for
instance.
\item Some way to access operating system facilities. Reading and writing
files would be useful, for instance, or accessing databases, or making
HTTP requests. Most general purpose languages provide some way to do this;
it would just be a matter of generating the appropriate code.
\item Better string manipulation. As it stands strings in WIG can't even
be indexed; the only supported operation is concatenation. Strong support
for string manipulation goes a long way towards making a language usable,
and again, it would simply be a matter of delegating to the target
language.
\end{itemize}

\subsection{Course Improvements}
On the whole, COMP 520 is a very fun and well structured course. The
workload is nontrivial, but the project is split up into clear, 
well-defined milestones, and making the source for the JOOS compiler
available provides ample guidance -- learning by example really is quite
effective. The only downside is the WIG language itself; it might have 
been an interesting concept back in the late 90s, but as it stands, it's 
awfully dated and obsolete. Writing anything of value with it is 
difficult. Modern web frameworks are both easier to use and more powerful.
As such, completing the project is less satisfying than it could be, since
it's likely {\tt wigismo} will never be used again. Of course, the same
could be said of any sort of coursework; these projects are undertaken
mainly for their instructive value, and little else. With this concept,
though -- building a compiler for a DSL -- there is such potential for
building something useful, and it seems silly to have students write
4500-line CGI script generators instead!

(One simple idea would be to flip the course around; provide the source for
a WIG compiler, and have students write a JOOS compiler. Generating code
that can actually run on a JVM is certainly more interesting than CGI!)

\subsection{Goodbye}
Ismail thinks compilers are fun and interesting and is taking COMP 621 
next semester, and applying for a Master's degree in this area. Carla's 
passions lie elsewhere.:w

\end{document}
